%%%%%%%%%%%%%%%%%%%%%%%%%%%%%%%%%%%%%%%%%%%%%%%%%%%%%%%%%%
%% Info sur latex
%% Olivier Morel
%% inspired by
%% template from reds institute for syslog
%%%%%%%%%%%%%%%%%%%%%%%%%%%%%%%%%%%%%%%%%%%%%%%%%%%%%%%%%%

% Paramétrage du papier utilisé
\documentclass[a4paper,12pt,oneside]{article}

% Ne pas toucher, nécessaire à la mise en forme
\usepackage[utf8x]{inputenc} 
\usepackage[T1]{fontenc} % Nouvelle norme pour codage des caractères
\usepackage{lmodern} % Nouvelle forme de la fonte ComputerModern
\usepackage[french]{babel} % Règles typographiques françaises, césures
\usepackage{graphicx} % Insertion images
\usepackage{float}  % allow to force the placement of picture with 'H'
\usepackage{epstopdf} % Convertir les images eps vers pdf
\usepackage[top=3cm, bottom=3cm, left=3cm, right=3cm, showframe=false]{geometry}
\usepackage{amsmath}
%\usepackage{hyperref}  % used to produce hyperlink when referencing but those links are marked by big red square on pdf softwares (in theory not when printed)
\usepackage{pdfpages}   % allow to include multipages pdf in document
\usepackage{color}  % allow to use colors
\usepackage{multirow}
\usepackage{multicol}
\usepackage{lastpage}
\usepackage[load-configurations = abbreviations]{siunitx}	% also load some abbreviations
\usepackage{tabularx}
\usepackage{hyperref}

\epstopdfsetup{outdir=./Images/converted_to_pdf/} % Dossier de sortie des images converties
\pagenumbering{arabic}
%\renewcommand{\familydefault}{\sfdefault}	% change font to helvetica but can cause warning on some problems
\sisetup{
	locale = FR,
	round-mode = figures,
	round-precision = 4,				% 4 digits in total
	scientific-notation = engineering,	% exponnant by multiple of 3
	per-mode			= symbol,
	per-symbol			= /,
	exponent-to-prefix  = true,
	output-decimal-marker = {.},		% decimal marker will always be printed as a '.'
	output-complex-root = j,			% imaginary unit will always be 'j'
	exponent-product = \cdot,			% exponant product will be a dot
	detect-all							% garde la même fonte que le reste du texte
}
\hypersetup{
	colorlinks=true,
	linkcolor=black,
	filecolor=black,
	urlcolor=blue,
}

\title{Introduction à \LaTeX}
\author{Olivier \bsc{Morel}}
\date{\today}

% Entête et pieds de page (A ADAPTER)
\pagestyle{empty}
\usepackage{fancyhdr}
\setlength{\headheight}{27.06pt}
\pagestyle{fancy}
% Entête :
\renewcommand{\headrulewidth}{1pt}
\fancyhead[L]{Introduction à \LaTeX}
% Pied de page
\fancyfoot{} % clear all footer fields
\renewcommand{\footrulewidth}{1pt}
\fancyfoot[R]{Page \thepage / \pageref{LastPage}} 
\fancyfoot[L]{Olivier \bsc{Morel}}



\begin{document}

\maketitle

% enlève le numéro sur la premiere page
	\thispagestyle{empty}
	\setcounter{page}{1}
	\newpage

\tableofcontents
\clearpage
	
\input{chapitres/Introduction.tex}
\section{Siunitx}
\subsection{Introduction}
Le package \emph{siunitx} aide le rédacteur à écrire des valeurs numériques et des unités simplement et de manière consistante.
Il offre des possibilités de configurations très complexes.
Les exemples proposés ici sont spécifiques à la configuration de siunitx trouvée dans les templates sur \url{https://github.com/morel-olivier/latex-templates}

\subsection{Usage basique}
\begin{tabularx}{0.8\textwidth}{ l l l }
	\hline
	action	& commande	& résultat	\\
	\hline
	écrire un nombre	& \verb|\num{3.14}|	& \num{3.14}	\\
	\hline
	\multirow{5}{40mm}{écrire une unité}	& \verb|\si{\m}|		& \si{\m}\\
											& \verb|\si{\km}|		& \si{\km}\\
											& \verb|\si{\kg}|		& \si{\kg}\\
											& \verb|\si{\per\s}|	& \si{\per\s}\\
											& \verb|\si{\m\per\kg}|	& \si{\m\per\kg}\\
	\hline
	\multirow{3}{40mm}{écrire un résultat}	& \verb|\SI{1.234e14}{\mol}|		& \SI{1.234e14}{\mol}\\
											& \verb|\SI{1.23456}{\mol}|		& \SI{1.23456}{\mol}\\
											& \verb|\SI{123456}{\mol}|		& \SI{123456}{\mol}\\
	\hline
	écrire un angle		& \verb|\ang{13.346}|	& \ang{13.346}\\
	\hline
	écrire une valeur \(\pm\) 	& \verb|\SI{+-3.145}{\volt}| & \SI{+-3.145}{\volt}	\\
	\hline
\end{tabularx}

Attention aux changements de minuscules et majuscule.


\end{document}