%%%%%%%%%%%%%%%%%%%%%%%%%%%%%%%%%%%%%%%%%%%%%%%%%%%%%%%%%%
%% Info sur latex
%% Olivier Morel
%% inspired by
%% template from reds institute for syslog
%%%%%%%%%%%%%%%%%%%%%%%%%%%%%%%%%%%%%%%%%%%%%%%%%%%%%%%%%%

% Paramétrage du papier utilisé
\documentclass[a4paper,12pt,oneside]{article}

% Ne pas toucher, nécessaire à la mise en forme
\usepackage[utf8x]{inputenc} 
\usepackage[T1]{fontenc} % Nouvelle norme pour codage des caractères
\usepackage{lmodern} % Nouvelle forme de la fonte ComputerModern
\usepackage[french]{babel} % Règles typographiques françaises, césures
\usepackage{graphicx} % Insertion images
\usepackage{float}  % allow to force the placement of picture with 'H'
\usepackage{epstopdf} % Convertir les images eps vers pdf
\usepackage[top=3cm, bottom=3cm, left=3cm, right=3cm, showframe=false]{geometry}
\usepackage{amsmath}
%\usepackage{hyperref}  % used to produce hyperlink when referencing but those links are marked by big red square on pdf softwares (in theory not when printed)
\usepackage{pdfpages}   % allow to include multipages pdf in document
\usepackage{color}  % allow to use colors
\usepackage{multirow}
\usepackage{multicol}
\usepackage{lastpage}
\usepackage[load-configurations = abbreviations]{siunitx}	% also load some abbreviations
\usepackage{tabularx}
\usepackage{hyperref}

\epstopdfsetup{outdir=./Images/converted_to_pdf/} % Dossier de sortie des images converties
\pagenumbering{arabic}
%\renewcommand{\familydefault}{\sfdefault}	% change font to helvetica but can cause warning on some problems
\sisetup{
	locale = FR,
	round-mode = figures,
	round-precision = 4,				% 4 digits in total
	scientific-notation = engineering,	% exponnant by multiple of 3
	per-mode			= symbol,
	per-symbol			= /,
	exponent-to-prefix  = true,
	output-decimal-marker = {.},		% decimal marker will always be printed as a '.'
	output-complex-root = j,			% imaginary unit will always be 'j'
	exponent-product = \cdot,			% exponant product will be a dot
	detect-all							% garde la même fonte que le reste du texte
}
\hypersetup{
	colorlinks=true,
	linkcolor=black,
	filecolor=black,
	urlcolor=blue,
}

\title{Introduction à \LaTeX}
\author{Olivier \bsc{Morel}}
\date{\today}

% Entête et pieds de page (A ADAPTER)
\pagestyle{empty}
\usepackage{fancyhdr}
\setlength{\headheight}{27.06pt}
\pagestyle{fancy}
% Entête :
\renewcommand{\headrulewidth}{1pt}
\fancyhead[L]{Introduction à \LaTeX}
% Pied de page
\fancyfoot{} % clear all footer fields
\renewcommand{\footrulewidth}{1pt}
\fancyfoot[R]{Page \thepage / \pageref{LastPage}} 
\fancyfoot[L]{Olivier \bsc{Morel}}



\begin{document}

\maketitle

% enlève le numéro sur la premiere page
	\thispagestyle{empty}
	\setcounter{page}{1}
	\newpage

\tableofcontents
\clearpage
	
\input{chapitres/Introduction.tex}
\section{Liens utiles}
\url{https://fr.overleaf.com/learn}

Trés bon site d'aide.
Toutefois, il ne vas pas toujour dans les détails trés techniques.

\url{https://github.com/Wookai/paper-tips-and-tricks}

Excellents document sur différentes combines pour se faciliter la vie et améliorer la qualité des documents.

\url{https://garsia.math.yorku.ca/~zabrocki/latexpanel/mathaccents.html}

Comment utiliser des équations dans l'environnement math.

\url{https://www.mathcha.io/}

Editeur de dessin, formule et autre qui permet d'exporter du \LaTeX.
\section{Siunitx}
\subsection{Introduction}
Le package \emph{siunitx} aide le rédacteur à écrire des valeurs numériques et des unités simplement et de manière consistante.
Il offre des possibilités de configurations très complexes.
Les exemples proposés ici sont spécifiques à la configuration de siunitx trouvée dans les templates sur \url{https://github.com/morel-olivier/latex-templates}

\subsection{Usage basique}
\begin{tabularx}{0.8\textwidth}{ l l l }
	\hline
	action	& commande	& résultat	\\
	\hline
	écrire un nombre	& \verb|\num{3.14}|	& \num{3.14}	\\
	\hline
	\multirow{5}{40mm}{écrire une unité}	& \verb|\si{\m}|		& \si{\m}\\
											& \verb|\si{\km}|		& \si{\km}\\
											& \verb|\si{\kg}|		& \si{\kg}\\
											& \verb|\si{\per\s}|	& \si{\per\s}\\
											& \verb|\si{\m\per\kg}|	& \si{\m\per\kg}\\
	\hline
	\multirow{3}{40mm}{écrire un résultat}	& \verb|\SI{1.234e14}{\mol}|		& \SI{1.234e14}{\mol}\\
											& \verb|\SI{1.23456}{\mol}|		& \SI{1.23456}{\mol}\\
											& \verb|\SI{123456}{\mol}|		& \SI{123456}{\mol}\\
	\hline
	écrire un angle		& \verb|\ang{13.346}|	& \ang{13.346}\\
	\hline
	écrire une valeur \(\pm\) 	& \verb|\SI{+-3.145}{\volt}| & \SI{+-3.145}{\volt}	\\
	\hline
\end{tabularx}

Attention aux changements de minuscules et majuscule.

\section{Inclure du code python dans \LaTeX}
\subsection{Introduction}
Inclure du python dans un document permet de faire les calculs directement dans le document.
Cela permet de ne pas avoir besoin de copier les nouveaux résultats des calculs à chaque correction des calculs.
De plus, cela permet de générer directement des graphiques dans \LaTeX.
Le package utilisé dans ce cas est \href{https://www.ctan.org/pkg/pythontex}{pythontex}.
Il est démontré comment l'installer et l'utiliser sur VS Code avec l'extension latex workshop

\subsection{Installation}
\subsection{Windows 10}

\subsection{Linux}
L'installation a été testée sur Ubuntu 21.10.

Premièrement, il faut vérifier si python est bien installé.
Pour ce faire, il faut entrer qu'une de ces commandes donne un résultat.
\begin{minted}{bash}
	$ python2 -V
	Python 2.7.18
	$ python3 -V
	Python 3.9.7
\end{minted}

Si ce n'est pas le cas, installez python avec la commande :
\begin{minted}{bash}
	$ sudo apt install python3
\end{minted}
Il semblerait que certaine librairie puisse ne pas être compatible avec python3 dans ce cas, vous pouvez installer python2.

Pour l'Utilisation avec VS Code, il faut préciser quel python doit être appelé par défaut.
Vérifiez ensuite que vous avez un python qui peut être appelé par défaut :
\begin{minted}{bash}
	$ sudo update-alternatives --list python
	update-alternatives: error: no alternatives for python
\end{minted}

Dans ce cas, entrez la commande suivante :
\begin{minted}[breaklines=true]{bash}
	$ sudo update-alternatives --install /usr/bin/python python /usr/bin/python3 1
\end{minted}

Vous avez maintenant une version de python par défaut que vous pouvez vérifier avec :
\begin{minted}{bash}
	$ python -V
	Python 3.9.7
\end{minted}

\subsection{Configuration de VS Code}
Afin de pouvoir utiliser pythontex, il faut encore créer une règle pour que VSCode compile correctement.
Pour ce faire il faut modifier le fichier settings.json qui contient touts les réglages de VS code.
Pour l'ouvrir, utiliser le racourci CTRL + MAJ + p.
Ensuite taper : "Settings: Open Settings (JSON).
Sous latex-workshop.latex.tools ajouter le code du Listing \ref{mint:configpythontex}.

\begin{listing}[H]
	\begin{minted}{json}
{
	"name": "pythontex",
	"command": "pythontex",
	"args": [
		"%DOC%"
	],
	"env": {}
},
	\end{minted}
	\caption{Configuration de pythonrex}
	\label{mint:configpythontex}
\end{listing}

Vérifiez ensuite qu'il existe bien une configuration pour pdflatex.
Sinon ajoutez le code du Listing \ref{mint:configpdflatex}.

\begin{listing}[H]
	\begin{minted}{json}
{
	"name": "pdflatex",
	"command": "pdflatex",
	"args": [
		"-synctex=1",
		"-interaction=nonstopmode",
		"--shell-escape",
		"-file-line-error",
		"%DOC%"
	],
	"env": {}
},
	\end{minted}
	\caption{Configuration de pdflatex}
	\label{mint:configpdflatex}
\end{listing}

Enfin aller sous "latex-workshop.latex.recipes" et ajouter le code du Listing \ref{}.

\begin{listing}[H]
	\begin{minted}{json}
{
	"name": "pdflatex -> pythontex -> pdflatex",
	"tools": [
		"pdflatex",
		"pythontex",
		"pdflatex"
	]
},
	\end{minted}
	\caption{Configuration de la recette}
	\label{mint:recette}
\end{listing}

\subsection{Utilisation}

\end{document}