\section{Siunitx}
\subsection{Introduction}
Le package \emph{siunitx} aide le rédacteur à écrire des valeurs numériques et des unités simplement et de manière consistante.
Il offre des possibilités de configurations très complexes.
La documentation complète se trouve sur \url{https://ctan.org/pkg/siunitx}.

\subsection{Configuration}
La configuration proposée ici écrit les nombres en notation ingénieur avec 4 chiffres de précision et des exposants au lieux des puissances de dix si il y a une unité.

\begin{listing}[H]
	\begin{minted}{tex}
\sisetup{
	locale = FR,
	round-mode = figures,
	% 4 digits in total
	round-precision = 4,
	% exponnant by multiple of 3
	scientific-notation = engineering,
	per-mode = symbol,
	per-symbol = /,
	exponent-to-prefix = true,
	% decimal marker will always be printed as a '.'
	output-decimal-marker = {.},		
	% imaginary unit will always be 'j'
	output-complex-root = j,			
	% exponant product will be a dot
	exponent-product = \cdot,			
	% keep same police as all the text
	detect-all							
}
	\end{minted}
	\caption{Configuration de siunitx}
	\label{mint:configSiunitx}
\end{listing}


\subsection{Usage basique}
Les exemples proposés ici sont spécifiques à la configuration de siunitx proposée ci-dessus (Listing \ref{mint:configSiunitx}).

\begin{center}
	\begin{adjustbox}{angle=90}
		\begin{tabular}{ l l l }
			\hline
			action	& commande	& résultat	\\
			\hline
			écrire un nombre	& \mintinline{latex}|\num{3.14}|	& \num{3.14}	\\
			notation simple 	& \mintinline{latex}|\num[scientific-notation=fixed]{0.1}|	& \num[scientific-notation = fixed]{0.1}	\\
			\hline
			\multirow{5}{40mm}{écrire une unité}	& \mintinline{latex}|\si{\m}|		& \si{\m}\\
													& \mintinline{latex}|\si{\km}|		& \si{\km}\\
													& \mintinline{latex}|\si{\kg}|		& \si{\kg}\\
													& \mintinline{latex}|\si{\per\s}|	& \si{\per\s}\\
													& \mintinline{latex}|\si{\m\per\kg}|	& \si{\m\per\kg}\\
			\hline
			\multirow{3}{40mm}{écrire un résultat}	& \mintinline{latex}|\SI{1.234e14}{\mol}|		& \SI{1.234e14}{\mol}\\
													& \mintinline{latex}|\SI{1.23456}{\mol}|		& \SI{1.23456}{\mol}\\
													& \mintinline{latex}|\SI{123456}{\mol}|		& \SI{123456}{\mol}\\
			\hline
			écrire un angle		& \mintinline{latex}|\ang{13.346}|	& \ang{13.346}\\
			\hline
			écrire une valeur \(\pm\) 	& \mintinline{latex}|\SI{+-3.145}{\volt}| & \SI{+-3.145}{\volt}	\\
			\hline
		\end{tabular}
	\end{adjustbox}
\end{center}

