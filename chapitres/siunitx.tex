\section{Siunitx}
\subsection{Introduction}
Le package \emph{siunitx} aide le rédacteur à écrire des valeurs numériques et des unités simplement et de manière consistante.
Il offre des possibilités de configurations très complexes.
Les exemples proposés ici sont spécifiques à la configuration de siunitx trouvée dans les templates sur \url{https://github.com/morel-olivier/latex-templates}

\subsection{Usage basique}
\begin{tabularx}{0.8\textwidth}{ l l l }
	\hline
	action	& commande	& résultat	\\
	\hline
	écrire un nombre	& \verb|\num{3.14}|	& \num{3.14}	\\
	\hline
	\multirow{5}{40mm}{écrire une unité}	& \verb|\si{\m}|		& \si{\m}\\
											& \verb|\si{\km}|		& \si{\km}\\
											& \verb|\si{\kg}|		& \si{\kg}\\
											& \verb|\si{\per\s}|	& \si{\per\s}\\
											& \verb|\si{\m\per\kg}|	& \si{\m\per\kg}\\
	\hline
	\multirow{3}{40mm}{écrire un résultat}	& \verb|\SI{1.234e14}{\mol}|		& \SI{1.234e14}{\mol}\\
											& \verb|\SI{1.23456}{\mol}|		& \SI{1.23456}{\mol}\\
											& \verb|\SI{123456}{\mol}|		& \SI{123456}{\mol}\\
	\hline
	écrire un angle		& \verb|\ang{13.346}|	& \ang{13.346}\\
	\hline
	écrire une valeur \(\pm\) 	& \verb|\SI{+-3.145}{\volt}| & \SI{+-3.145}{\volt}	\\
	\hline
\end{tabularx}

Attention aux changements de minuscules et majuscule.
