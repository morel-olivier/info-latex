\section{Inclure du code python dans \LaTeX}
\subsection{Introduction}
Inclure du python dans un document permet de faire les calculs directement dans le document.
Cela permet de ne pas avoir besoin de copier les nouveaux résultats des calculs à chaque correction des calculs.
De plus, cela permet de générer directement des graphiques dans \LaTeX.
Le package utilisé dans ce cas est \href{https://www.ctan.org/pkg/pythontex}{pythontex}.
Il est démontré comment l'installer et l'utiliser sur VS Code avec l'extension latex workshop

\subsection{Installation}
\subsection{Windows 10}

\subsection{Linux}
L'installation a été testée sur Ubuntu 21.10.

Premièrement, il faut vérifier si python est bien installé.
Pour ce faire, il faut entrer qu'une de ces commandes donne un résultat.
\begin{minted}{bash}
	$ python2 -V
	Python 2.7.18
	$ python3 -V
	Python 3.9.7
\end{minted}

Si ce n'est pas le cas, installez python avec la commande :
\begin{minted}{bash}
	$ sudo apt install python3
\end{minted}
Il semblerait que certaine librairie puisse ne pas être compatible avec python3 dans ce cas, vous pouvez installer python2.

Pour l'Utilisation avec VS Code, il faut préciser quel python doit être appelé par défaut.
Vérifiez ensuite que vous avez un python qui peut être appelé par défaut :
\begin{minted}{bash}
	$ sudo update-alternatives --list python
	update-alternatives: error: no alternatives for python
\end{minted}

Dans ce cas, entrez la commande suivante :
\begin{minted}[breaklines=true]{bash}
	$ sudo update-alternatives --install /usr/bin/python python /usr/bin/python3 1
\end{minted}

Vous avez maintenant une version de python par défaut que vous pouvez vérifier avec :
\begin{minted}{bash}
	$ python -V
	Python 3.9.7
\end{minted}

\subsection{Configuration de VS Code}
Afin de pouvoir utiliser pythontex, il faut encore créer une règle pour que VSCode compile correctement.
Pour ce faire il faut modifier le fichier settings.json qui contient touts les réglages de VS code.
Pour l'ouvrir, utiliser le racourci CTRL + MAJ + p.
Ensuite taper : "Settings: Open Settings (JSON).
Sous latex-workshop.latex.tools ajouter le code du Listing \ref{mint:configpythontex}.

\begin{listing}[H]
	\begin{minted}{json}
{
	"name": "pythontex",
	"command": "pythontex",
	"args": [
		"%DOC%"
	],
	"env": {}
},
	\end{minted}
	\caption{Configuration de pythonrex}
	\label{mint:configpythontex}
\end{listing}

Vérifiez ensuite qu'il existe bien une configuration pour pdflatex.
Sinon ajoutez le code du Listing \ref{mint:configpdflatex}.

\begin{listing}[H]
	\begin{minted}{json}
{
	"name": "pdflatex",
	"command": "pdflatex",
	"args": [
		"-synctex=1",
		"-interaction=nonstopmode",
		"--shell-escape",
		"-file-line-error",
		"%DOC%"
	],
	"env": {}
},
	\end{minted}
	\caption{Configuration de pdflatex}
	\label{mint:configpdflatex}
\end{listing}

Enfin aller sous "latex-workshop.latex.recipes" et ajouter le code du Listing \ref{}.

\begin{listing}[H]
	\begin{minted}{json}
{
	"name": "pdflatex -> pythontex -> pdflatex",
	"tools": [
		"pdflatex",
		"pythontex",
		"pdflatex"
	]
},
	\end{minted}
	\caption{Configuration de la recette}
	\label{mint:recette}
\end{listing}

\subsection{Utilisation}